% download_install.tex

%%%%%%%%%%%%%%%%%%%%%%%%%%%%%%%%%%%%%%%%%%%%%%%%%%%%%%%%%%%%
\SUBSECTION{Download and install \pmclib}
\label{sec:install_pmclib}
%%%%%%%%%%%%%%%%%%%%%%%%%%%%%%%%%%%%%%%%%%%%%%%%%%%%%%%%%%%%


The package \pmclib\ can be downloaded from the \CosmoPMC\ site
%\url{http://www2.iap.fr/users/kilbinge/CosmoPMC}.
\url{http://www.cosmopmc.info}.

After downloading, unpack the gzipped tar archive
%
\command{tar xzf pmclib\_x.y.tar.gz}
%
This creates the \pmclib\ root directory \direc{pmclib\_x.y}.
\pmclib\ uses \progr{waf}\footnote{\url{http://code.google.com/p/waf}}
instead of configure/make to compile and build the software. Change to
that directory and type
%
\command{./waf --local configure}
%
See \progr{./waf --help} for options. The packages \progr{lua},
\progr{hdf5} and \progr{lapack} are optionally linked with \pmclib\
but are not necessary to run \CosmoPMC. Corresponding warnings of
missing files can be ignored. Instead of a local installation
(indicated by \progr{--local}), a install prefix can be specified
with \progr{--prefix=PREFIX} (default \direc{/usr/local}).

%%% Patch 1.1_1.01

%Finally, run
%
%\command{./waf build install}
%
%to build the \pmclib\ dynamic libraries.
%See
%\url{http://www2.iap.fr/users/kilbinge/CosmoPMC/pmclib} for more
%details about the PMC library.


%%%%%%%%%%%%%%%%%%%%%%%%%%%%%%%%%%%%%%%%%%%%%%%%%%%%%%%%%%%%
\SUBSECTION{Patch \pmclib}
%%%%%%%%%%%%%%%%%%%%%%%%%%%%%%%%%%%%%%%%%%%%%%%%%%%%%%%%%%%%

For \CosmoPMC\ v $\ge 1.2$ and pmclib v1.x, a patch of the
latter is necessary. From \url{http://www.cosmopmc.info} , download
\file{patch\_pmclib\_1.x\_1.2.tar.gz} and follow the instructions in
the readme file \file{readme\_patch\_pmclib\_1.x\_1.2.txt}.


%%%%%%%%%%%%%%%%%%%%%%%%%%%%%%%%%%%%%%%%%%%%%%%%%%%%%%%%%%%%
\SUBSECTION{Download and install \CosmoPMC}
%%%%%%%%%%%%%%%%%%%%%%%%%%%%%%%%%%%%%%%%%%%%%%%%%%%%%%%%%%%%

The newest version of \CosmoPMC\ can be downloaded from the site
\url{http://www.cosmopmc.info}.

First, unpack the gzipped tar archive
%
\command{tar xzf CosmoPMC\_v\CosmoPMCVersion.tar.gz}
%
This creates the the \CosmoPMC\ root directory
\direc{CosmoPMC\_v\CosmoPMCVersion}. Change to that directory and run
%
\command{[python] ./configure.py}
%
This (poor man's) configure script copies the file
\file{Makefile.no\_host} to \file{Makefile.host} and sets
host-specific variables and flags as given by the command-line
arguments. For a complete list, see `\progr{configure.py --help}'.

Alternatively, you can copy by hand the file
\file{Makefile.no\_host} to \file{Makefile.host} and edit it. If the
flags in this file are not sufficient to successfully compile the code,
you can add more flags by rerunning \progr{configure.py}, or by
manually editing \file{Makefile.main}. Note that a flag in
\file{Makefile.main} is overwritten if the same flag is present in
\file{Makefile.host}.


To compile the code, run
%
\command{make; make clean}
%
On success, symbolic links to the binary executables (in \direc{./exec}) will be
set in \direc{./bin}.

It is convenient to define the environment variable \envvar{COSMOPMC}
and to set it to the main
\CosmoPMC\ directory. For example, in the C-shell:
%
\command{setenv COSMOPMC /path/to/CosmoPMC\_v\CosmoPMCVersion}
%
This command can be placed into the startup file
(e.g.~\file{\symbol{126}/.cshrc} for the C-shell).
One can also add \direc{\COSMOPMCDIR/bin} to
the \envvar{PATH} environment variable.

